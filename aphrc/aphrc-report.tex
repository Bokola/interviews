% Options for packages loaded elsewhere
% Options for packages loaded elsewhere
\PassOptionsToPackage{unicode}{hyperref}
\PassOptionsToPackage{hyphens}{url}
\PassOptionsToPackage{dvipsnames,svgnames,x11names}{xcolor}
%
\documentclass[
  letterpaper,
  DIV=11,
  numbers=noendperiod]{scrartcl}
\usepackage{xcolor}
\usepackage[margin=1in]{geometry}
\usepackage{amsmath,amssymb}
\setcounter{secnumdepth}{5}
\usepackage{iftex}
\ifPDFTeX
  \usepackage[T1]{fontenc}
  \usepackage[utf8]{inputenc}
  \usepackage{textcomp} % provide euro and other symbols
\else % if luatex or xetex
  \usepackage{unicode-math} % this also loads fontspec
  \defaultfontfeatures{Scale=MatchLowercase}
  \defaultfontfeatures[\rmfamily]{Ligatures=TeX,Scale=1}
\fi
\usepackage{lmodern}
\ifPDFTeX\else
  % xetex/luatex font selection
\fi
% Use upquote if available, for straight quotes in verbatim environments
\IfFileExists{upquote.sty}{\usepackage{upquote}}{}
\IfFileExists{microtype.sty}{% use microtype if available
  \usepackage[]{microtype}
  \UseMicrotypeSet[protrusion]{basicmath} % disable protrusion for tt fonts
}{}
\makeatletter
\@ifundefined{KOMAClassName}{% if non-KOMA class
  \IfFileExists{parskip.sty}{%
    \usepackage{parskip}
  }{% else
    \setlength{\parindent}{0pt}
    \setlength{\parskip}{6pt plus 2pt minus 1pt}}
}{% if KOMA class
  \KOMAoptions{parskip=half}}
\makeatother
% Make \paragraph and \subparagraph free-standing
\makeatletter
\ifx\paragraph\undefined\else
  \let\oldparagraph\paragraph
  \renewcommand{\paragraph}{
    \@ifstar
      \xxxParagraphStar
      \xxxParagraphNoStar
  }
  \newcommand{\xxxParagraphStar}[1]{\oldparagraph*{#1}\mbox{}}
  \newcommand{\xxxParagraphNoStar}[1]{\oldparagraph{#1}\mbox{}}
\fi
\ifx\subparagraph\undefined\else
  \let\oldsubparagraph\subparagraph
  \renewcommand{\subparagraph}{
    \@ifstar
      \xxxSubParagraphStar
      \xxxSubParagraphNoStar
  }
  \newcommand{\xxxSubParagraphStar}[1]{\oldsubparagraph*{#1}\mbox{}}
  \newcommand{\xxxSubParagraphNoStar}[1]{\oldsubparagraph{#1}\mbox{}}
\fi
\makeatother


\usepackage{longtable,booktabs,array}
\usepackage{calc} % for calculating minipage widths
% Correct order of tables after \paragraph or \subparagraph
\usepackage{etoolbox}
\makeatletter
\patchcmd\longtable{\par}{\if@noskipsec\mbox{}\fi\par}{}{}
\makeatother
% Allow footnotes in longtable head/foot
\IfFileExists{footnotehyper.sty}{\usepackage{footnotehyper}}{\usepackage{footnote}}
\makesavenoteenv{longtable}
\usepackage{graphicx}
\makeatletter
\newsavebox\pandoc@box
\newcommand*\pandocbounded[1]{% scales image to fit in text height/width
  \sbox\pandoc@box{#1}%
  \Gscale@div\@tempa{\textheight}{\dimexpr\ht\pandoc@box+\dp\pandoc@box\relax}%
  \Gscale@div\@tempb{\linewidth}{\wd\pandoc@box}%
  \ifdim\@tempb\p@<\@tempa\p@\let\@tempa\@tempb\fi% select the smaller of both
  \ifdim\@tempa\p@<\p@\scalebox{\@tempa}{\usebox\pandoc@box}%
  \else\usebox{\pandoc@box}%
  \fi%
}
% Set default figure placement to htbp
\def\fps@figure{htbp}
\makeatother


% definitions for citeproc citations
\NewDocumentCommand\citeproctext{}{}
\NewDocumentCommand\citeproc{mm}{%
  \begingroup\def\citeproctext{#2}\cite{#1}\endgroup}
\makeatletter
 % allow citations to break across lines
 \let\@cite@ofmt\@firstofone
 % avoid brackets around text for \cite:
 \def\@biblabel#1{}
 \def\@cite#1#2{{#1\if@tempswa , #2\fi}}
\makeatother
\newlength{\cslhangindent}
\setlength{\cslhangindent}{1.5em}
\newlength{\csllabelwidth}
\setlength{\csllabelwidth}{3em}
\newenvironment{CSLReferences}[2] % #1 hanging-indent, #2 entry-spacing
 {\begin{list}{}{%
  \setlength{\itemindent}{0pt}
  \setlength{\leftmargin}{0pt}
  \setlength{\parsep}{0pt}
  % turn on hanging indent if param 1 is 1
  \ifodd #1
   \setlength{\leftmargin}{\cslhangindent}
   \setlength{\itemindent}{-1\cslhangindent}
  \fi
  % set entry spacing
  \setlength{\itemsep}{#2\baselineskip}}}
 {\end{list}}
\usepackage{calc}
\newcommand{\CSLBlock}[1]{\hfill\break\parbox[t]{\linewidth}{\strut\ignorespaces#1\strut}}
\newcommand{\CSLLeftMargin}[1]{\parbox[t]{\csllabelwidth}{\strut#1\strut}}
\newcommand{\CSLRightInline}[1]{\parbox[t]{\linewidth - \csllabelwidth}{\strut#1\strut}}
\newcommand{\CSLIndent}[1]{\hspace{\cslhangindent}#1}



\setlength{\emergencystretch}{3em} % prevent overfull lines

\providecommand{\tightlist}{%
  \setlength{\itemsep}{0pt}\setlength{\parskip}{0pt}}



 


\usepackage{booktabs}
\usepackage{longtable}
\usepackage{array}
\usepackage{multirow}
\usepackage{wrapfig}
\usepackage{float}
\usepackage{colortbl}
\usepackage{pdflscape}
\usepackage{tabu}
\usepackage{threeparttable}
\usepackage{threeparttablex}
\usepackage[normalem]{ulem}
\usepackage{makecell}
\usepackage{xcolor}
\KOMAoption{captions}{tableheading}
\usepackage{sectsty}
\usepackage{xcolor}
\usepackage{fancyhdr}
\definecolor{wdlblue}{HTML}{012D4A}
\definecolor{wdlteal}{HTML}{00A7B5}
\sectionfont{\color{wdlblue}\uppercase}
\subsectionfont{\color{wdlblue}}
\usepackage{etoolbox}
\makeatletter
\patchcmd{\@sect}{\fi#7}{\fi\color{wdlteal}\hrule height 1.5pt \kern 3pt\color{black}#7}{}{}
\makeatother
\pagestyle{fancy}
\makeatletter
\@ifpackageloaded{caption}{}{\usepackage{caption}}
\AtBeginDocument{%
\ifdefined\contentsname
  \renewcommand*\contentsname{Table of contents}
\else
  \newcommand\contentsname{Table of contents}
\fi
\ifdefined\listfigurename
  \renewcommand*\listfigurename{List of Figures}
\else
  \newcommand\listfigurename{List of Figures}
\fi
\ifdefined\listtablename
  \renewcommand*\listtablename{List of Tables}
\else
  \newcommand\listtablename{List of Tables}
\fi
\ifdefined\figurename
  \renewcommand*\figurename{Figure}
\else
  \newcommand\figurename{Figure}
\fi
\ifdefined\tablename
  \renewcommand*\tablename{Table}
\else
  \newcommand\tablename{Table}
\fi
}
\@ifpackageloaded{float}{}{\usepackage{float}}
\floatstyle{ruled}
\@ifundefined{c@chapter}{\newfloat{codelisting}{h}{lop}}{\newfloat{codelisting}{h}{lop}[chapter]}
\floatname{codelisting}{Listing}
\newcommand*\listoflistings{\listof{codelisting}{List of Listings}}
\makeatother
\makeatletter
\makeatother
\makeatletter
\@ifpackageloaded{caption}{}{\usepackage{caption}}
\@ifpackageloaded{subcaption}{}{\usepackage{subcaption}}
\makeatother
\usepackage{bookmark}
\IfFileExists{xurl.sty}{\usepackage{xurl}}{} % add URL line breaks if available
\urlstyle{same}
\hypersetup{
  pdftitle={Predicting New COVID-19 Cases in the US: A Comparison of a Compartmental Model (SIR) and XGBoost Machine Learning Model},
  pdfauthor={Basil Okola},
  colorlinks=true,
  linkcolor={blue},
  filecolor={Maroon},
  citecolor={Blue},
  urlcolor={Blue},
  pdfcreator={LaTeX via pandoc}}


\title{Predicting New COVID-19 Cases in the US: A Comparison of a
Compartmental Model (SIR) and XGBoost Machine Learning Model}
\author{Basil Okola}
\date{February 11, 2026}
\begin{document}
\maketitle
\begin{abstract}
\textbf{Background}: Figuring out the number of new COVID-19 cases was
key in aligning interventions to hotspots. This included ensuring there
were enough beds for admissions, key protection gear like face masks,
oxygen supply and intensive care unit space. A lot of mechanistic models
were build borrowing from the traditional SIR model. Additionally,
researchers explored machine learning techniques to predict the course
of the infections.

\textbf{Methods}: In this report, we compare the performance of a SIR
model to an XGBOOST prediction model in predicting new cases in each of
the American states, 3 years after the pandemic.

\textbf{Results}: The XGBOOST ML approach demonstrated robust
goodness-of-fit, effectively capturing the non-linear peak dynamics
across varying geographic replicates.
\end{abstract}


\section{Introduction}\label{introduction}

We chose to use data on COVID-19 cases assembled by a team from
\href{https://vac-lshtm.shinyapps.io/ncov_tracker/\#}{The London School
of Health \& Tropical Medicine} who built a global COVID-19 tracker
shiny app. This was purely due to convenience and the short turn around
time for the assessment as there was an already ongoing personal project
that modularized parts of the said app into a golem framework and so it
was easier adding a prediction modelling capabilities rather than
starting a new project. The
\href{https://github.com/Bokola/COVID19Dash}{COVID19Dash} app can be
installed from github and run seamlessly. For predictions look at the
\texttt{Prediction\ Model} tab.

\section{Methods}\label{methods}

\subsection{Data}\label{data}

We used \texttt{cv\_states} data that is part of the datasets in the
\texttt{COVID19Dash} package. It compiles daily COVID-19 cases in each
of the states in the US. It also comes with limited spatial support
(longitudes \& latitudes). We chose \texttt{new\_cases} as the outcome
variable and generated rolling means for 7, 14, and 30 days as
predictors in addition to the frequency of each state in the data to
avoid having to encode the categorical variable.

\subsection{Models}\label{models}

We fitted a SIR model - a compartmental mechanistic framework that
simulates the spread of infectious diseases by transitioning a
population through susceptible, infectious, and recovered states using
differential equations. It relies on the interaction between the
transmission rate \(\beta\) and the recovery rate \(\gamma\) to
determine the velocity and peak of an epidemic curve\textsuperscript{1}.

\subsubsection{SIR framework}\label{sir-framework}

The mechanistic sir model assumes a fixed population \(N\) where
individuals transition between susceptible (\(S\)), infectious (\(I\)),
and recovered (\(R\)) compartments. the dynamics are governed by the
following system of ordinary differential equations, which were
implemented using the \texttt{desolve} package in our workflow:

\[\begin{aligned}
\frac{dS}{dt} &= -\beta \frac{S I}{N} \\
\frac{dI}{dt} &= \beta \frac{S I}{N} - \gamma I \\
\frac{dR}{dt} &= \gamma I
\end{aligned}\]

where \(\beta\) represents the effective transmission rate and
\(\gamma\) denotes the removal or recovery rate. individuals are assumed
to move from the susceptible to the infectious compartment at a rate
proportional to the product of \(S\) and \(I\).

A critical threshold in this modeling framework is the basic
reproduction number (\(R_0\)), which defines the average number of
secondary infections produced by a single infected individual in a
completely susceptible population. in our analysis, we calculated
\(R_0\) using the optimized parameters from each state:

\[R_0 = \frac{\beta}{\gamma}\]

An epidemic persists only when \(R_0 > 1\). our model fitting used an
optimization routine to minimize the residuals between the predicted
\(I(t)\) compartment and the observed case ratios across the 50 american
states.

\subsubsection{XGBOOST model}\label{xgboost-model}

We also fitted an XGBOOST model to predict the new cases. XGBoost, or
Extreme Gradient Boosting, is an ensemble decision‐tree algorithm, like
random forest regressions, but able to model more complex interactions
due to its ability to boost individual trees and does not rely on a
single tree. It uses a scalable tree‐boosting system to optimize
predictions\textsuperscript{2}.

\subsection{Missing data}\label{missing-data}

We did not encounter missing data. However, depending on the objective
of the research, there are a number of pathways available. In a purely
ML prediction setting, researchers apply simple methods like mean/median
for numeric variables or mode for categorical variables. In a purely
inferential task, multiple imputations are employed under missing at
random (MAR) assumption and a sensitivity analysis done to verify that
the imputation method is not driving model decisions.

\subsection{Accounting for spatial and tempororal
structure}\label{accounting-for-spatial-and-tempororal-structure}

We did not work with a hierarchical spatial and temporal variation.
However, one approach is through Hierarchical Bayesian models. In R this
can be accomplished through R-INLA\textsuperscript{3}. This involves
decomposing observed variance into structured and unstructured random
effects across multiple levels, using neighboring data to inform
estimates in sparse regions.

\section{Results}\label{results}

\subsection{Model comparisons}\label{model-comparisons}

The XGBOOST performed better than the simple SIR model in predicting new
cases.

\pandocbounded{\includegraphics[keepaspectratio]{aphrc-report_files/figure-pdf/unnamed-chunk-2-1.pdf}}

\section{Discussions}\label{discussions}

We have build two models and compared their performance using visual
plots and other metrics logged to the shiny app. We did not however
undertake a simulation due to time constraints. We consulted AI on
figuring out bugs from XGBOOST which we found to be a challenge running
in R environment.

\section{Reproducibility}\label{reproducibility}

Install the \href{https://github.com/Bokola/COVID19Dash}{COVID19Dash}
using \texttt{devtools::install\_github("bokola/COVID19Dash")} command.
You can then run the app interactively by calling \texttt{run\_app()}.
To reproduce the contents of this document, ensure the other packages
called in \texttt{setup} chunk are installed.

\section*{Reference}\label{reference}
\addcontentsline{toc}{section}{Reference}

\phantomsection\label{refs}
\begin{CSLReferences}{0}{1}
\bibitem[\citeproctext]{ref-ross1916application}
\CSLLeftMargin{1 }%
\CSLRightInline{Ross R. An application of the theory of probabilities to
the study of a priori pathometry.---part i. \emph{Proceedings of the
Royal Society of London Series A, Containing papers of a mathematical
and physical character} 1916; \textbf{92}: 204--30.}

\bibitem[\citeproctext]{ref-chen2016xgboost}
\CSLLeftMargin{2 }%
\CSLRightInline{Chen T. XGBoost: A scalable tree boosting system.
\emph{Cornell University} 2016.}

\bibitem[\citeproctext]{ref-moraga2019geospatial}
\CSLLeftMargin{3 }%
\CSLRightInline{Moraga P. Geospatial health data: Modeling and
visualization with r-INLA and shiny. Chapman; Hall/CRC, 2019.}

\end{CSLReferences}




\end{document}
